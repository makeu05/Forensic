\documentclass[a4paper,12pt]{report}
\usepackage[utf8]{inputenc}
\usepackage[T1]{fontenc}
\usepackage[french]{babel}
\usepackage[top=2.5cm,bottom=2.5cm,left=2.5cm,right=2.5cm]{geometry}
\usepackage{graphicx}
\usepackage{amsmath, amssymb}
\usepackage{booktabs}
\usepackage{array}
\usepackage{setspace}
\usepackage{hyperref}
\usepackage{xcolor}
\usepackage{libertine}
\hypersetup{
	colorlinks=true,
	linkcolor=blue,
	urlcolor=blue,
	citecolor=black
}
\setstretch{1.2}
\setlength{\parindent}{0cm}
\setlength{\parskip}{1ex plus 0.5ex minus 0.2ex}
\newcommand{\hsp}{\hspace{20pt}}
\newcommand{\HRule}{\rule{\linewidth}{0.5mm}}

\title{Rapport de configuration d'une infrastructure réseau fonctionnelle}
\author{Belva MAKEU TENKU}
\date{Octobre 2025}

\begin{document}

% ---------------- Page de titre ----------------
\begin{titlepage}
	\begin{sffamily}
		\begin{center}
			\includegraphics[scale=0.04]{logo_polytech.JPG}~\\[1.5cm]
			\textsc{\LARGE ÉCOLE NATIONALE SUPÉRIEURE POLYTECHNIQUE DE YAOUNDÉ}\\[2cm]
			\textsc{\Large Département de Génie Informatique}\\[2cm]
			\textsc{\large Introduction aux Techniques d'Investigation Numérique}\\[1.5cm]
			\HRule \\[0.4cm]
			{ \huge \bfseries Rapport de configuration d'une infrastructure réseau fonctionnelle\\[0.4cm] }
			\HRule \\[2cm]
			\includegraphics[scale=0.2]{logo_polytech.JPG} \\[2cm]
			\begin{minipage}{0.4\textwidth}
				\begin{flushleft} \large
					MAKEU TENKU STELY BELVA\\
					CIN-4\\
				\end{flushleft}
			\end{minipage}
			\begin{minipage}{0.4\textwidth}
				\begin{flushright} \large
					\emph{Superviseur :} M. \textsc{Thierry Minka}\\
				\end{flushright}
			\end{minipage}
			\vfill
		\end{center}
	\end{sffamily}
\end{titlepage}
\tableofcontents
\newpage

% --- Introduction ---
\chapter*{Introduction}
\addcontentsline{toc}{chapter}{Introduction}
Ce premier laboratoire a pour objectif de configurer une infrastructure réseau complète, fonctionnelle et sécurisée. 
L’environnement comprend un routeur en frontière, une zone démilitarisée (DMZ) hébergeant un serveur web sous Linux, et un réseau local (LAN) 
contenant un poste de travail Windows. Cette mise en place vise à fournir une base pour des activités d’investigation numérique simulant une entreprise victime d’un ransomware.

% --- Objectifs du Lab ---
\chapter{Objectif du Lab}
Ce laboratoire permet de :
\begin{itemize}
    \item Configurer un réseau fonctionnel comprenant un équipement de frontière, un LAN et une DMZ ;
    \item Mettre en place un serveur web accessible depuis l’intérieur et l’extérieur du réseau ;
    \item Configurer les adresses IP, les interfaces réseau et les politiques de sécurité ;
    \item Tester la connectivité et la disponibilité des ressources.
\end{itemize}

% --- Topologie et description ---
\chapter{Topologie et Description du Réseau}
\section{Schéma logique}
\noindent
Le réseau comporte trois segments principaux : un réseau externe (Internet), une DMZ, et un réseau local (LAN). 
Le schéma ci-dessous illustre l’architecture mise en œuvre.

\begin{figure}[h!]
    \centering
    \includegraphics[width=0.9\textwidth]{schema_lab1.png}
    \caption{Topologie du réseau dans GNS3}
\end{figure}

\section{Description des composants}
\begin{itemize}
    \item \textbf{Routeur :} assure la connectivité entre le réseau externe et le réseau interne ;
    \item \textbf{Switch manageable :} permet la segmentation en sous-réseaux et la communication inter-réseaux ;
    \item \textbf{Pare-feu :} contrôle les flux entre le LAN et la DMZ ;
    \item \textbf{Serveur Linux :} héberge une application web ;
    \item \textbf{Poste Windows :} simule un utilisateur interne.
\end{itemize}

% --- Matériel et logiciels ---
\chapter{Matériel et Logiciels Utilisés}

\begin{table}[h!]
\centering
\begin{tabular}{|m{4cm}|m{5cm}|m{2cm}|m{3cm}|}
\hline
\textbf{Composant} & \textbf{Logiciel / Matériel} & \textbf{Version} & \textbf{Rôle} \\ \hline
Routeur & Cisco IOL  & - & Équipement de frontière \\ \hline
Pare-feu & FortiGate  & - & Sécurisation de la DMZ \\ \hline
Switch & Switch manageable IOL & - & Communication inter-réseaux \\ \hline
Machine virtuelle Windows & Windows 10 & - & Poste client LAN \\ \hline
Machine virtuelle Linux & Ubuntu / Kali linux & - & Serveur Web \\ \hline
Logiciel de virtualisation & VMware & - & Hébergement des VM \\ \hline
Simulateur réseau & GNS3 & - & Simulation de l’infrastructure \\ \hline
\end{tabular}
\caption{Matériel et logiciels utilisés dans le Lab}
\end{table}

% --- Étapes de réalisation ---
\chapter{Étapes de Réalisation}

\section{Création des Machines Virtuelles}
\subsection*{Machine Windows 10}
\begin{itemize}
    \item Disque dur : 10 Go
    \item RAM : 2 Go
    \item Données copiées : 2 Go de fichiers variés
    \item Rôle : Poste de travail utilisateur sur le LAN
\end{itemize}

\subsection*{Machine Linux (serveur web)}
\begin{itemize}
    \item Disque dur : 10 Go
    \item RAM : 2 Go
    \item Distribution : Ubuntu 
    \item Application web déployée via Django
\end{itemize}

\section{Création de l’Infrastructure dans GNS3}
\begin{enumerate}
    \item Installation de GNS3 et création du projet \texttt{LAB1}.
    \item Ajout d’un routeur (équipement de frontière).
    \item Configuration des interfaces :
    \begin{itemize}
        \item \texttt{R-Eth0} : Adresse publique (Internet)
        \item \texttt{R-Eth1} : Adresse privée (LAN)
    \end{itemize}
    \item Ajout d’un switch manageable connecté au routeur.
    \item Configuration :
    \begin{itemize}
        \item \texttt{S-Eth0} connecté à \texttt{R-Eth1}
        \item \texttt{S-Eth1} connecté au pare-feu (\texttt{F-Eth0})
        \item \texttt{S-Eth2} connecté au poste Windows
    \end{itemize}
    \item Ajout d’un pare-feu :
    \begin{itemize}
        \item \texttt{F-Eth0} : vers le switch (DMZ)
        \item \texttt{F-Eth1} : vers le serveur web Linux
    \end{itemize}
\end{enumerate}

\section{Tests de Connectivité}
\begin{itemize}
    \item Ping entre le poste Windows et le serveur Linux ;
    \item Accès HTTP à l’application web depuis le LAN ;
    \item Vérification du trafic à travers le pare-feu ;
    \item Test de connectivité externe (Internet simulé).
\end{itemize}

% --- Résultats ---
\chapter{Résultats et Vérifications}
Les tests effectués confirment la fonctionnalité de l’infrastructure :
\begin{itemize}
    \item Le poste Windows accède correctement à l’application web du serveur Linux ;
    \item Le pare-feu filtre les flux entre le LAN et la DMZ ;
    \item Le routage et les interfaces sont opérationnels ;
    \item L'application est fonctionnel sur le serveur web.
\end{itemize}

\begin{figure}
    \centering
    \includegraphics[width=1\linewidth]{info_kali.png}
    \caption{configuration de kali}
    \label{fig:placeholder}
\end{figure}
\begin{figure}
    \centering
    \includegraphics[width=1\linewidth]{interface_pare_feu.png}
    \caption{interface du parefeu}
    \label{fig:placeholder}
\end{figure}
\begin{figure}
    \centering
    \includegraphics[width=1\linewidth]{policy_pare_feu.png}
    \caption{policy du parefeu}
    \label{fig:placeholder}
\end{figure}
\begin{figure}
    \centering
    \includegraphics[width=1\linewidth]{serveur_ubuntu.png}
    \caption{serveur ubuntu}
    \label{fig:placeholder}
\end{figure}
\begin{figure}
    \centering
    \includegraphics[width=1\linewidth]{app_kali.png}
    \caption{application sur kali linux}
    \label{fig:placeholder}
\end{figure}
\begin{figure}
        \centering
        \includegraphics[width=0.90\linewidth]{image.png}
        \caption{Accès à l’application web depuis le poste Windows}
        \label{fig:placeholder}
    \end{figure}
\begin{figure}[h!]
    \centering
    \includegraphics[width=0.85\textwidth]{test_web_access.png}
    \caption{ping vers le serveur web}
\end{figure}



% --- Conclusion ---
\chapter*{Conclusion}
\addcontentsline{toc}{chapter}{Conclusion}
Ce laboratoire a permis de mettre en œuvre un environnement réseau complet, intégrant les notions de routage, segmentation, sécurisation et virtualisation. 
Cette base pourra être utilisée pour les travaux d’investigation numérique à venir, notamment la simulation d’attaques et l’analyse post-incident.

\end{document}