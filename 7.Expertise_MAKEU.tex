\documentclass[a4paper,12pt]{report}
\usepackage[utf8]{inputenc}
\usepackage[T1]{fontenc}
\usepackage[french]{babel}
\usepackage[top=2.5cm,bottom=2.5cm,left=2.5cm,right=2.5cm]{geometry}
\usepackage{graphicx}
\usepackage{amsmath, amssymb}
\usepackage{booktabs}
\usepackage{array}
\usepackage{setspace}
\usepackage{hyperref}
\usepackage{xcolor}
\usepackage{libertine}
\hypersetup{
	colorlinks=true,
	linkcolor=blue,
	urlcolor=blue,
	citecolor=black
}
\setstretch{1.2}
\setlength{\parindent}{0cm}
\setlength{\parskip}{1ex plus 0.5ex minus 0.2ex}
\newcommand{\hsp}{\hspace{20pt}}
\newcommand{\HRule}{\rule{\linewidth}{0.5mm}}

\title{Rapport d'Investigation Numérique (OSINT)}
\author{Belva MAKEU TENKU}
\date{Octobre 2025}

\begin{document}

% ---------------- Page de titre ----------------
\begin{titlepage}
	\begin{sffamily}
		\begin{center}
			\includegraphics[scale=0.04]{logo_polytech.JPG}~\\[1.5cm]
			\textsc{\LARGE ÉCOLE NATIONALE SUPÉRIEURE POLYTECHNIQUE DE YAOUNDÉ}\\[2cm]
			\textsc{\Large Département de Génie Informatique}\\[2cm]
			\textsc{\large Introduction aux Techniques d'Investigation Numérique}\\[1.5cm]
			\HRule \\[0.4cm]
			{ \huge \bfseries Rapport d'Investigation Numérique sur les hypothèses \\[0.4cm] }
			\HRule \\[2cm]
			\includegraphics[scale=0.2]{logo_polytech.JPG} \\[2cm]
			\begin{minipage}{0.4\textwidth}
				\begin{flushleft} \large
					MAKEU TENKU STELY BELVA\\
					CIN-4\\
				\end{flushleft}
			\end{minipage}
			\begin{minipage}{0.4\textwidth}
				\begin{flushright} \large
					\emph{Superviseur :} M. \textsc{Thierry Minka}\\
				\end{flushright}
			\end{minipage}
			\vfill
		\end{center}
	\end{sffamily}
\end{titlepage}

\section*{Introduction}
Dans le cadre de ce devoir d’investigation numérique, nous nous intéressons à l’analyse d’une affaire impliquant des éléments technologiques et des traces informatiques. L’objectif de ce travail est de formuler des hypothèses plausibles sur les événements survenus, en s’appuyant sur les indices numériques disponibles et sur les principes de la cybersécurité et de la criminalistique informatique.
Cette démarche s’inscrit dans une logique méthodique : observer, collecter, analyser et interpréter les traces afin de reconstituer les faits et de comprendre les responsabilités éventuelles.
Ainsi, à travers cette investigation, nous proposerons plusieurs hypothèses cohérentes permettant d’expliquer les circonstances de l’affaire, avant de les confronter aux éléments techniques et logiques recueillis.
\newpage
\section*{PRÉSENTATION DES HYPOTHÈSES}

\subsubsection{Hypothèse 1 : Suicide}
Hypothèse écartée, car la victime ne pouvait pas s’être elle-même infligé les sévices corporels observés avant sa mort.

\subsubsection{Hypothèse 2 : Le coupable n’appartient pas au groupe des personnes inculpées}
Hypothèse rejetée, puisque les dernières personnes présentes avec la victime au moment de sa mort font partie des inculpés. Aucun élément ne permet donc de supposer l’implication d’un tiers extérieur à l’affaire.

\subsubsection{Hypothèse 3 : Le coupable fait partie des personnes inculpées}
Hypothèse retenue, car le croisement des données de localisation et des témoignages confirme que certaines des personnes inculpées étaient bien les dernières à se trouver avec la victime avant son décès.






\end{document}