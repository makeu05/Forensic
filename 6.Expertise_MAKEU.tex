\documentclass[a4paper,12pt]{report}
\usepackage[utf8]{inputenc}
\usepackage[T1]{fontenc}
\usepackage[french]{babel}
\usepackage[top=2.5cm,bottom=2.5cm,left=2.5cm,right=2.5cm]{geometry}
\usepackage{graphicx}
\usepackage{amsmath, amssymb}
\usepackage{booktabs}
\usepackage{array}
\usepackage{setspace}
\usepackage{hyperref}
\usepackage{xcolor}
\usepackage{libertine}
\hypersetup{
	colorlinks=true,
	linkcolor=blue,
	urlcolor=blue,
	citecolor=black
}
\setstretch{1.2}
\setlength{\parindent}{0cm}
\setlength{\parskip}{1ex plus 0.5ex minus 0.2ex}
\newcommand{\hsp}{\hspace{20pt}}
\newcommand{\HRule}{\rule{\linewidth}{0.5mm}}

\title{Rapport d'Investigation Numérique (OSINT)}
\author{Belva MAKEU TENKU}
\date{Octobre 2025}

\begin{document}

% ---------------- Page de titre ----------------
\begin{titlepage}
	\begin{sffamily}
		\begin{center}
			\includegraphics[scale=0.04]{logo_polytech.JPG}~\\[1.5cm]
			\textsc{\LARGE ÉCOLE NATIONALE SUPÉRIEURE POLYTECHNIQUE DE YAOUNDÉ}\\[2cm]
			\textsc{\Large Département de Génie Informatique}\\[2cm]
			\textsc{\large Introduction aux Techniques d'Investigation Numérique}\\[1.5cm]
			\HRule \\[0.4cm]
			{ \huge \bfseries Rapport d'Investigation Numérique D'un document\\[0.4cm] }
			\HRule \\[2cm]
			\includegraphics[scale=0.2]{logo_polytech.JPG} \\[2cm]
			\begin{minipage}{0.4\textwidth}
				\begin{flushleft} \large
					MAKEU TENKU STELY BELVA\\
					CIN-4\\
				\end{flushleft}
			\end{minipage}
			\begin{minipage}{0.4\textwidth}
				\begin{flushright} \large
					\emph{Superviseur :} M. \textsc{Thierry Minka}\\
				\end{flushright}
			\end{minipage}
			\vfill
		\end{center}
	\end{sffamily}
\end{titlepage}


\section{INTRODUCTION}
Ce rapport contient des éléments d'analyse d'un document fournit par M. Thierry MINKA. Étant un document confidentiel , il sera utilisé des alias(numéros) ici pour nommer les individus concernés. Il s'agit d'une affaire d'investigation numérique où il faut identifier les éléments de l'investigateur qui ont permis de sortir ce document.La victime sera nommée en utilisant le numéro 0.
\newpage
\section{ELEMENTS TROUVÉS}
\begin{itemize}
    \item Des données de géolocalisation et la fiche technique du véhicule utilisé. 
    \item  données des appels sur le téléphone de la victime 0
    \item appel téléphonique et WhatsApp 
    \item données téléphoniques du numéro 8
    \item listing d'appels téléphoniques et données de localisation
    \item fiche de localisation de la victime
    \item image de vidéosurveillances du bureau
    \item bande sonore 
    \item image de vidéo surveillance urbaine de Yaoundé 
    \item thermocopie de la capture d'écran 
\end{itemize}




\end{document}